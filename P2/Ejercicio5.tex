\documentclass[11pt,a4paper]{article}
\usepackage[utf8]{inputenc}
\usepackage[english]{babel}	%Idioma
\usepackage{amsmath}
\usepackage{amsfonts}
\usepackage{amssymb}
\usepackage{graphicx} 	%Añadir imágenes
\usepackage{geometry}	%Ajustar márgenes
\usepackage[export]{adjustbox}[2011/08/13]
\usepackage{float}
\restylefloat{table}
\usepackage[hidelinks]{hyperref}
\usepackage{titling}
\graphicspath{{/home/nazaret/Escritorio/LaTEX}}
%\usepackage{minted}
\usepackage{multirow}
\usepackage{caption}
\usepackage{multicol}
\usepackage{array}
\selectlanguage{english}

%Opciones de encabezado y pie de página:
\usepackage{fancyhdr}
\pagestyle{fancy}
\lhead{}
\rhead{}
\lfoot{Fundamentos de Redes}
\cfoot{}
\rfoot{\thepage}
\renewcommand{\headrulewidth}{0.4pt}
\renewcommand{\footrulewidth}{0.4pt}

%Opciones de fuente:
\usepackage[utf8]{inputenc}
\usepackage[default]{sourcesanspro}
\usepackage{sourcecodepro}
\usepackage[T1]{fontenc}

\setlength{\parindent}{15pt}
\setlength{\headheight}{15pt}
\setlength{\voffset}{10mm}

% Custom colors
\usepackage{color}
\definecolor{deepblue}{rgb}{0,0,0.5}
\definecolor{deepred}{rgb}{0.6,0,0}
\definecolor{deepgreen}{rgb}{0,0.5,0}

\usepackage{listings}

% Diagrama de estados
\usepackage{tikz}
\usetikzlibrary{automata,positioning,arrows}

% Tablas con saltos de línea en las celdas
\usepackage{fourier}
\usepackage{makecell}

\renewcommand\theadalign{bc}
\renewcommand\theadfont{\bfseries}
\renewcommand\theadgape{\Gape[4pt]}
\renewcommand\cellgape{\Gape[4pt]}

\begin{document}
\begin{titlepage}

\begin{minipage}{\textwidth}

\centering
\includegraphics[width=0.5\textwidth]{/home/nazaret/Escritorio/LaTEX/logo.png}\\

\textsc{\Large Fundamentos de Redes\\[0.2cm]}
\textsc{GRADO EN INGENIERÍA INFORMÁTICA}\\[1cm]

{\Huge\bfseries Práctica 2\\}
\noindent\rule[-1ex]{\textwidth}{3pt}\\[3.5ex]
{\large\bfseries Ejercicio 5}
\end{minipage}

\vspace{1.5cm}
\begin{minipage}{\textwidth}
\centering

\textbf{Autores}\\ {Javier Gálvez Obispo\\Nazaret Román Guerrero}\\[2.5ex]
\includegraphics[width=0.3\textwidth]{/home/nazaret/Escritorio/LaTEX/etsiit.jpeg}\\[0.1cm]
\vspace{1cm}
\textsc{Escuela Técnica Superior de Ingenierías Informática y de Telecomunicación}\\
\vspace{1cm}
\textsc{Curso 2018-2019}
\end{minipage}
\end{titlepage}

\pagenumbering{gobble}
\pagenumbering{arabic}
\tableofcontents
\thispagestyle{empty}

\newpage

\section{Descripción de la aplicación, funcionalidad y actores que intervienen}

La aplicación diseñada sirve para jugar al clásico juego \textit{Batalla Naval}, donde dos jugadores intentan hundir la flota del adversario y alzarse victoriosos.\\

En esta aplicación, el jugador humano se enfrenta al servidor, utilizando el protocolo \textsc{\textbf{TCP}}, que es usado para iniciar la comunicación entre cliente y servidor.\\

Los actores son los siguientes:\\
	\begin{itemize}
	\item El servidor, que es el oponente a derrotar por el cliente. Se encarga de establecer una conexión cuando le llega la petición del cliente.
	\item El cliente, el jugador humano que inicia la comunicación con el servidor y que intentará derrotar al servidor.
	\item El procesador, que es el que lleva a cabo toda la funcionalidad, especialmente la del servidor (ya que realmente éste solo se encarga de llevar a cabo la comunicación y delega todo el juego al procesador).
	\end{itemize}

\section{Diagrama de estados del servidor}

\begin{figure}[H]
\centering
\begin{tikzpicture}[->,>=stealth',shorten >=1pt,auto,node distance=5cm,scale = 1,transform shape]
	\node[state] (start) {$start$}; 
	\node[state] (play) [right=of start] {$play$}; 
	\node[state] (end) [below=of start] {$end$};
	
	\draw 	(start) edge[above] node{(conex/pos y valor)} (play)
			(start) edge[left] node{(me rindo/has perdido)} (end)
			(play) edge[right] node{(me rindo/has perdido), (*/has ganado), (*/has perdido)} (end)
			(play) edge[loop right] node{(pos y valor/pos y valor)} (play);
\end{tikzpicture}
\end{figure}

\newpage

\section{Mensajes que intervienen}

\begin{itemize}
	\item Mensajes envíados por el cliente
	\begin{figure}[H]
	\centering
	\begin{tabular}{|l|l|l|}
		\hline
		\thead{Código} & \thead{Cuerpo} & \thead{Descripción}\\ \hline
		100 & \textsc{\makecell{nueva posicion +\\valor posicion cliente +\\ anterior posicion}} & \makecell{Este mensaje lo envía el cliente al servidor para\\indicar la nueva posición que el cliente quiere\\saber y la respuesta a la posición que el\\servidor le ha enviado anteriormente} \\ \hline
	\end{tabular}
	\end{figure}
	
	\item Mensajes envíados por el servidor
	\begin{figure}[H]
	\centering
	\begin{tabular}{|l|l|l|}
		\hline
		\thead{Código} & \thead{Cuerpo} & \thead{Descripción}\\ \hline
		200 & \textsc{\makecell{nueva posicion +\\valor posicion servidor +\\ anterior posicion}} & \makecell{Este mensaje lo envía el servidor al cliente para\\indicar la nueva posición que el servidor quiere\\saber (es aleatoria) y la respuesta a la posición\\que el cliente le envía en el turno anterior} \\ \hline
	\end{tabular}
	\end{figure}
\end{itemize}

\section{Evaluación de la aplicación}

\begin{itemize}
	\item Inicio del juego: se muestran los tableros del jugador (el terminal izquierdo, donde se muestra a la izquierda el tablero donde se apunta qué posiciones ha preguntado el servidor, y a la derecha el tablero del jugador) y el servidor (el terminal de la derecha).
	
	\begin{figure}[H]
	\centering
	\includegraphics[scale=0.21]{/home/nazaret/Escritorio/1.png}
	\end{figure}
	
	\item Partida: a medida que el juego avanza, los tableros van rellenándose. Cuando el servidor toca (o hunde) un barco, la letra \textit{b} es sustituida por una \textit{x}. Las casillas que han sido visitadas pero en las que solo había agua se coloca una almohadilla.\\
	
	En el otro tablero se rellena con \textit{a} en el caso de que en el tablero del servidor hubiera agua, y con \textit{b} si hubiera un barco. 

	\begin{figure}[H]
	\centering
	\includegraphics[scale=0.21]{/home/nazaret/Escritorio/2.png}
	\end{figure}

	\item El cliente se rinde: cuando el jugador quiera rendirse, solo ha de introducir la letra \textit{\textbf{r}}, de forma que la partida finaliza, tal y como se observa en la siguiente imagen:
	
	\begin{figure}[H]
	\centering
	\includegraphics[scale=0.21]{/home/nazaret/Escritorio/3.png}
	\end{figure}
\end{itemize}
\end{document}